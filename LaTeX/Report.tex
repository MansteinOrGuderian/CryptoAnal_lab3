\chapter{}
\section{Вступні відомості}
\noindent\textbf{Мета роботи:} Ознайомлення з підходами побудови атак на асиметричні криптосистеми на прикладі атак на криптосистему 
RSA, а саме атаки на основі китайської теореми про лишки, що є успішною при використанні однакового малого значення відкритої експоненти 
для багатьох користувачів та атаки "зустріч посередині"{}, яка можлива у випадку, якщо шифротекст є невеликим числом, що є добутком двох чисел.

\noindent\textbf{Постановка задачі:}
\begin{enumerate}
    \item Створіть репозиторій у системі контролю версій Git/GitHub;
    \item Реалізувати атаку з малою експонентою на основі китайської теореми про лишки.
    \item Реалізувати атаку "зустріч посередині"{} та порівняти її швидкодію з повним перебором можливих відкритих текстів.
    \item Оформити звiт до комп’ютерного практикуму.
\end{enumerate}
\section{Результати виконання роботи. Варіант 15}

% \begin{figure}[!ht]
%     \centering
%     \begin{minipage}{0.9\linewidth}
%         \includegraphics[width=0.9\textwidth, scale=1.2]{ReportPic/report_4.png}
%     \end{minipage}
% \end{figure}

\section{Висновки:}
